\chapter{Expenditure Minimization}

Recall that in our discussion of firm theory, we noted that the firm's problem of profit maximization is the dual problem of cost minimization. That is, if we were to determine the set of inputs that maximized profits, we would get the same result if we knew this maximum profit level and then determined the set of inputs that minimized the costs in order to achieve that level of profit. 

Analogously, in the case of consumer theory, the dual problem of utility maximization is known as \vocab{expenditure minimization.} While utility maximization asks for the set of goods needed to maximize utility under some constraint (e.g. what is the most utility I can achieve with \$20?), expenditure minimization asks for the set of goods that most cheaply attains a given level of utility (e.g. what is the cheapest way to achieve 100 utils of utility?). This chapter will study the expenditure minimization problem and explore how examining the utility maximization and expenditure minimization problems together gives us a fuller picture of how consumer behavior reacts to prices.

\section{Problem Setup}

Formally, suppose we have goods $1, \ldots, n$ with prices $p_1, \ldots, p_n$, respectively, and our choice variables $x_1, \ldots, x_n$ correspond to how much we purchase of each good, respectively. Denote our utility function from these goods as $u(x_1, \ldots, x_n)$. Suppose we must attain a utility level of at least $\bar{u}$, where $\bar{u}$ is some exogenous variable that is given to us. What is the cheapest way to achieve $\bar{u}$? We can write down our expenditure minimization problem as 
$$\min _{x_{1}, x_{2}, \ldots, x_{n}} \sum_{i=1}^{n} p_{i} x_{i} \text { s.t. } u\left(x_{1}, x_{2}, \ldots, x_{n}\right) \geq \bar{u}.$$
If we use vector notation and denote $\vec{p}$ as our prices and $\vec{x}$ as our chosen quantities, we can write this more succinctly as
$$\min _{\vec{x} \geq 0} \vec{p} \cdot \vec{x} \text { s.t. } u(\vec{x}) \geq \bar{u}$$

We can solve this problem similar to how we solved the cost minimization problem from our study of firm theory.

\subsection*{First Order Conditions}
\TODO{Standardize whether we add or subtract the Lagrangian term}

We can write our Lagrangian in vector form as
$$\Lagr(\mathbf{x}, \lambda)=\mathbf{p} \cdot \mathbf{x} - \lambda[u(\mathbf{x}) - \bar{u}].$$
Our first order conditions can be written as 
$$
\begin{cases}
\Lagr_{x_{i}}=p_{i}-\lambda^{*} u_{x_{i}}\left(\mathbf{x}^{*}\right)=0 \text { for all } i \\
\Lagr_{\lambda}=u\left(\mathbf{x}^{*}\right) - \bar{u}=0.
\end{cases}.$$
Notice how this looks similar to our conditions from utility maximization. The last condition is simply the requirement that we attain $\bar{u}$. Our first set of conditions regarding each $\Lagr_{x_{i}}$ term is the same as what we had from utility maximization, except since our constraint is different, the values of $\lambda^*$ and $\vec{x}^*$ are different that what they were in the utility maximization case. Note that we could also have written this first set of conditions in vector form as
$$\mathbf{p}=\lambda^{*} \nabla u\left(\mathbf{x}^{*}\right)$$

Recognize that our first order conditions give us $n+1$ linear equations with $n+1$ unknowns. We thus have a solution for $\lambda^*$ as well as solutions for
$$x_i^*(\vec{p}, \bar{u})$$
for $i = 1, 2, \ldots, n$. Notice that in the case of utility maximization, our solutions were of the form $x_i^*(\vec{p}, y)$. The important difference now is that our solution to the expenditure minimization problem is in terms of the required utility level rather than budget. Whereas we commonly denote $x_i^*(\vec{p}, y)$ as Marshallian demand, the solution to the utility maximization problem, we will denote the solution we just derived for the expenditure minimization problem as 
$$h_i(\vec{p}, \bar{u})$$
for $i=1, 2, \ldots, n$. This expression is known as the \vocab{Hicksian demand}, also referred to as the \vocab{compensated demand}. Moving forward, we will often drop arguments and simply write $x_i^*$ and $h_i^*$ for Marshallian and Hicksian demand, respectively, but it is important to remember how the arguments of these functions differ.

\section{Expenditure Function}

We now turn to the value function for the expenditure minimization problem, which is called the \vocab{expenditure function}:
$$e(\mathbf{p}, \bar{u})=\mathbf{p} \cdot \mathbf{h}(\mathbf{p}, \bar{u})=\sum_{i=1}^{n} p_{i} h_{i}(\mathbf{p}, \bar{u}).$$

In the last chapter, we noted that there are some similarities between a consumer's indirect utility function from utility maximization and a firm's profit function from profit maximization. However, the analogy was not exact. First, while the profit function is convex in prices, the indirect utility function is quasi-convex in prices. Second, while the profit function is homogeneous of degree 1, the indirect utility function is not.

With the expenditure function, we will notice that it is an exact analogy to a firm's cost function from cost minimization. We encourage the reader to refer back to Section~\ref{sec:cost_properties} to notice the parallels between properties of the cost function and the expenditure function.

\subsection*{Properties of the Expenditure Function}

\begin{description}
\item[Shephard's Lemma] $\pdv{e}{p_i}(\vec{p}, \bar{u}) = h_{i}(\mathbf{p}, \bar{u})$. In the case of cost minimization, we saw that the derivative of cost with respect to input prices was equal to the input demand. Analogously, in the case of expenditure minimization, the derivative of expenditure with respect to prices of goods is equal to the Hicksian demand of those goods. The lemma and the proof are the exact same as what we saw before.

\begin{proof}
Applying the constrained envelope theorem, we have
$$\pdv{e}{p_i}(\vec{p}, \bar{u}) = \frac{\partial}{\partial p_i}\left\{\mathbf{p} \cdot \mathbf{h}+\lambda^{*}[\bar{u}-u(\mathbf{h})]\right\} = h_{i}(\mathbf{p}, \bar{u}).$$
\end{proof}

\item[Homogeneous of degree 1 in prices] $e(\alpha\vec{p}, \bar{u}) = \alpha e(\vec{p}, \bar{u})$. The proof is the same as in Section~\ref{sec:cost_properties} and is encouraged as an exercise for the reader. The intuition is also the same: if we changed the units of the prices of all goods, then the units of our expenditure function would scale accordingly.

\item[Concave in prices] For all $\alpha \in [0, 1]$ and any price vectors $\vec{p}_1$ and $\vec{p}_2$, 
$$e(\alpha \vec{p}_1 + (1-\alpha)\vec{p}_2, \bar{u}) \geq \alpha e(\vec{p}_1, \bar{u}) + (1-\alpha) e(\vec{p}_2, \bar{u}).$$
Again, the proof is the same as in Section~\ref{sec:cost_properties} and is encouraged as an exercise for the reader. The basic idea of the proof is that at both $(\vec{p}_1, \bar{u})$ and $(\vec{p}_2, \bar{u})$, purchasing $\vec{h}(\alpha \vec{p}_1 + (1-\alpha)\vec{p}_2, \bar{u})$ will achieve the utility constraint, but you can spend weakly less by purchasing $\vec{h}(\vec{p}_1, \bar{u})$ or $\vec{h}(\vec{p}_2, \bar{u})$ at these points, respectively. A similar interpretation follows as well: as a consumer, you would rather have prices fluctuate than have them stay constant at their average value, since you can reoptimize your consumption bundle at each of the fluctuations to achieve a lower expenditure.

\item[Non-decreasing in prices and utility] If $\vec{p}$ of $\bar{u}$ increase, then $e(\vec{p}, \bar{u})$ weakly increases. The proof is straightforward: if prices or required utility fell from a previous value, then the previous bundle would still satisfy the constraint, so minimum expenditure cannot rise.

\item[Continuous] We will not prove this property of the expenditure function, but it is good to remember.

\end{description}

\TODO{Add example of Cobb Douglas}

\section{Hicksian and Marshallian Duality}
The Hicksian and Marshallian have offered us two different demand functions. However, the Hicksiand and the Marshallian are related in two important ways.
\begin{description}
\item[Given prices, the indirect utility function and the expenditure function are inverses.]
We observe that
$$v(\vec{p}, e(\vec{p}, \bar{u})) = \bar{u}.$$
Intuitively, we defined $e(\vec{p}, \bar{u})$ as the minimum cost to achieve $\bar{u}$, so if we have a starting budget of $e(\vec{p}, \bar{u})$, then the maximum utility we can achieve with this budget is $\bar{u}$.
Similarly, we observe that 
$$e(\vec{p}, v(\vec{p}, y)) = y.$$
This time, we defined $v(\vec{p}, y)$ to be the maximum utility we can achieve with budget $y$, so the minimum budget needed achieve utility level $v(\vec{p}, y)$ must be $y$. We conclude that if prices $\vec{p}$ are given, then the value functions $e$ and $v$ are inverses of each other.

\item[At optimality, Marshallian and Hicksian demand coincide.] 
We observe that 
$$\mathbf{x}(\mathbf{p}, e(\mathbf{p}, \bar{u}))=\mathbf{h}(\mathbf{p}, \bar{u}).$$
Intuitively, $\vec{h}(\vec{p}, \bar{u})$ is the cheapest bundle to attain utility $\bar{u}$, and it costs $e(\vec{p}, \bar{u})$. Thus, if we had a budget of $e(\vec{p}, \bar{u})$, then most satisfying bundle to buy should also be $\vec{h}(\vec{p}, \bar{u})$.
Similarly, we observe that
$$\mathbf{h}(\mathbf{p}, v(\mathbf{p}, y))=\mathbf{x}(\mathbf{p}, y).$$
Now, $\vec{x}(\vec{p}, y)$ is the most satisfying bundle to buy with a budget $y$, and it gives us utility $v(\mathbf{p}, y)$. Thus, if we must achieve a utility of $v(\mathbf{p}, y)$, then the cheapest bundle that attains this utility should also be $\vec{x}(\vec{p}, y)$. 

To summarize, since utility-maximization and expenditure-minimization both yield optimal consumption bundles, these bundles coincide when our budget from the utility-maximization problem is just enough to attain the required utility from the expenditure-minimization problem. 
\TODO{I left out the geometric interpretation because I think it's more for the interested reader than it is helpful for understanding, but I might add this back in?}
\end{description}

\section{Slutsky Equation} 
One of the questions are likely interested in now that we know how people demand goods is figuring out how demand changes for a given change in price. However, the answer is not always straightforward. Intuitively, if the price of a single good increases, we can separate the effect on consumption into two channels. First, that good becomes relatively more expensive and other goods become relatively cheaper; this effect is called the \vocab{substitution effect}. Second, the consumer has effectively become poorer in terms of the options she can afford; this effect is called the \vocab{income effect}. This section will study the \vocab{Slutsky equation}, which uses our ideas of Marshallian and Hicksian demand to derive an equation that contains both of these pieces together.

\subsection*{Derivation}
Recall from duality\TODO{Add reference} that at an optimum, the Marshallian and the Hicksian must be equal. Focusing on a single good $i$, this implies that, 
$$x_{i}(\mathbf{p}, e(\mathbf{p}, \bar{u}))=h_{i}(\mathbf{p}, \bar{u}),$$

By the Implicit Function Theorem, we can differentiate both sides of this equation with respect to $p_i$, yielding 
$$\frac{d}{d p_{i}} x_{i}(\mathbf{p}, e(\mathbf{p}, \bar{u}))=\frac{d}{d p_{i}} h_{i}(\mathbf{p}, \bar{u}).$$
Using the multivariate chain rule, this becomes 
$$\frac{\partial x_{i}}{\partial p_{i}}(\mathbf{p}, e(\mathbf{p}, \bar{u}))+\frac{\partial x_{i}}{\partial y}(\mathbf{p}, e(\mathbf{p}, \bar{u})) \frac{\partial e}{\partial p_{i}}(\mathbf{p}, \bar{u})=\frac{\partial h_{i}}{\partial p_{i}}(\mathbf{p}, \bar{u}).$$
By Shephard's Lemma, we know $\pdv{e}{p_i}(\mathbf{p}, \bar{u})=h_{i}(\mathbf{p}, \bar{u}),$ so we can simplify our earlier equation to 
$$\frac{\partial x_{i}}{\partial p_{i}}(\mathbf{p}, e(\mathbf{p}, \bar{u}))+h_{i}(\mathbf{p}, \bar{u}) \frac{\partial x_{i}}{\partial y}(\mathbf{p}, e(\mathbf{p}, \bar{u}))=\frac{\partial h_{i}}{\partial p_{i}}(\mathbf{p}, \bar{u}).$$

We now use the first fact that the expenditure function and the indirect utility function are inverses to rewrite this equation in terms of some budget $y$ instead of $\bar{u}$. Define $y = e(\vec{p}, \bar{u})$. Then $\bar{u} = v(p, y)$ by the first fact, and $h_i(\vec{p}, \bar{u}) = x_i(\vec{p}, y)$ by the second fact. We can substitute these terms into our equation to get 
$$\frac{\partial x_{i}}{\partial p_{i}}(\mathbf{p}, y)+x_{i}(\mathbf{p}, y) \frac{\partial x_{i}}{\partial y}(\mathbf{p}, y)=\frac{\partial h_{i}}{\partial p_{i}}(\mathbf{p}, v(\mathbf{p}, y)).$$
Rearranging these terms gives us the famous \vocab{Slutsky equation}
$$
\boxed{\frac{\partial x_{i}}{\partial p_{i}}(\mathbf{p}, y)=\frac{\partial h_{i}}{\partial p_{i}}(\mathbf{p}, v(\mathbf{p}, y))-x_{i}(\mathbf{p}, y) \frac{\partial x_{i}}{\partial y}(\mathbf{p}, y).}
$$
Suppressing the arguments, we can write this more succinctly as 
$$\frac{\partial x_{i}}{\partial p_{i}}=\frac{\partial h_{i}}{\partial p_{i}}-x_{i} \frac{\partial x_{i}}{\partial y}.$$

\subsection*{Interpretation}
The effect of price changes on consumption is exactly the $\frac{\partial x_{i}}{\partial p_{i}}$ term on the left, while the right hand side of the equation separates the substitution and income effects we were interested in. 

\begin{description}

\item[Substitution effect.] This component is given by the first term
$$\frac{\partial h_{i}}{\partial p_{i}}(\mathbf{p}, v(\mathbf{p}, y)).$$
Intuitively, this term represents the first-order change solely from good $i$ becoming more expensive relative to other goods, assuming that we could still afford the original consumption bundle. Another explanation is that this term represents the effect of a price change if real income is kept constant, where real income is defined in terms of the ideal price index discussion from the previous section.

We claim that the substitution effect is always negative; intuitively, if an item becomes relatively more expensive, the substitution effect would cause a shift in spending to the now relatively cheaper items. This fact is called the \vocab{Law of Compensated Demand.}
\begin{proof}
Recall that the expenditure function is concave in $\vec{p}$, so 
$$
\frac{\partial^{2} e}{\partial p_{i}^{2}}(\mathbf{p}, \bar{u})<0.
$$
By Shephard's Lemma, this means
$$
\frac{\partial}{\partial p_{i}} \frac{\partial e}{\partial p_{i}}(\mathbf{p}, \bar{u})=\frac{\partial h_{i}}{\partial p_{i}}(\mathbf{p}, \bar{u})<0.
$$
\end{proof}

\item[Income effect.] This component is given by the second term
$$-x_{i} \frac{\partial x_{i}}{\partial y}.$$
Intuitively, if the consumer becomes $\Delta y$ poorer, the consumption of good $i$ decreases by 
$$
\Delta x_{i} \approx \frac{\partial x_{i}}{\partial y} \Delta y.
$$
If the consumer was originally buying $x_i$ of good $i$, then a price increase of $\Delta p_i$ effectively makes her poorer by
$$
\Delta y \approx -x_{i} \Delta p_{i}.
$$
Putting these pieces together, the income effect of a small price change $\Delta p_i$ is 
$$
\Delta x_{i} \simeq-x_{i} \frac{\partial x_{i}}{\partial y} \Delta p_{i},
$$
which matches our result from Slutsky's equation.

What is the sign of the income effect? It depends! If the income effect is negative (i.e. if $\frac{\partial x_{i}}{\partial y}>0$) then $\frac{\partial x_{i}}{\partial p_{i}}<0$, since the substitution effect is always negative. This means that the demand function is downward sloping, as we might imagine intuitively. However, this might not always be the case; we might have a positive income effect when $\frac{\partial x_{i}}{\partial y}<0$ for some items (e.g. cheap dorm food) that we consume less when we become richer. 

\end{description} 

\subsection*{General Slutsky Equation}
In the original Slutsky equation, we were interested in the effect of a price change in good $i$ on the consumption of good $i$, so we differentiated with respect to $p_i$. If we were instead interested in the effect of a price change in some other good $j$ on the consumption of good $i$, we would differentiate with respect to $p_j$. The resulting derivation would look identical, and our final result would be the \vocab{general Slutsky equation}
$$
\boxed{\frac{\partial x_{i}}{\partial p_{j}}=\frac{\partial h_{i}}{\partial p_{j}}-x_{j} \frac{\partial x_{i}}{\partial y}.}
$$
The $-x_{j} \frac{\partial x_{i}}{\partial y}$ term represents the income effect, where the price of good $j$ makes the consumer poorer in real terms, which has some effect on the consumption of good $i$ depending on whether good $i$ is an inferior or normal good. However, whereas the substitution effect used to be unambiguously negative, the new substitution effect $\frac{\partial h_{i}}{\partial p_{j}}$ in the general equation can be positive or negative, depending on whether goods $i$ and $j$ are substitutes or complements, respectively.

\subsubsection*{Net and Gross Substitutes}
\TODO{Not sure if we want the separate definition and proposition environments in here}
Using this new substitution effect from the general Slutsky equation, we introduce two different ideas for how goods can be substitutes.

\begin{definition*}[Net substitutes]
We say that goods $i$ and $j$ are \vocab{net substitutes} if 
$$
\frac{\partial h_{i}}{\partial p_{j}}=\frac{\partial h_{j}}{\partial p_{i}}>0.
$$
\end{definition*}
Intuitively, two goods are net substitutes if an increase in the price of one good causes the consumption of the other good to rise, assuming that utility must be kept constant. We note that this definition is symmetric, which we prove in the following proposition.

\begin{proposition*}
Net substitutability is a symmetric condition. That is, $$\frac{\partial h_{i}}{\partial p_{j}}=\frac{\partial h_{j}}{\partial p_{i}}.$$ 
\end{proposition*}

\begin{proof}
By Shephard's Lemma, 
$$h_i(\vec{p}, \bar{u}) = \pdv{e(\vec{p}, \bar{u})}{p_i}.$$
Differentiating with respect to $p_j$ gives
$$\pdv{h_i(\vec{p}, \bar{u})}{p_j} = \frac{\partial^2 e(\vec{p}, \bar{u})}{\partial p_i \partial p_j}.$$
By Young's Theorem, we can swap the order of the partial derivatives, so 
$$\frac{\partial^2 e(\vec{p}, \bar{u})}{\partial p_i \partial p_j} = \frac{\partial^2 e(\vec{p}, \bar{u})}{\partial p_j \partial p_i}.$$
Thus, we conclude 
$$\pdv{h_i(\vec{p}, \bar{u})}{p_j} = \pdv{h_j(\vec{p}, \bar{u})}{p_i}$$
\end{proof}
We also claim that every good must have some net substitutes.
\begin{proposition*}
For all goods $j$, there exists good $i$ such that $\pdv{h_i}{p_j} > 0$.
\end{proposition*}
\begin{proof}
We know by the definition of the Hicksian that
$$u(\vec{h}(\vec{p}, \bar{u})) = \bar{u}.$$
Differentiating both sides by $p_j$ and using the multivariate chain rule yields 
$$
\sum_{i=1}^{n} u_{i} \frac{\partial h_{i}}{\partial p_{j}}=0.
$$
By the assumption of non-satiation (i.e. marginal utility for every good is always positive), $u_i > 0$ for all $i$. By the law of compensated demand, which we showed previously in this chapter, we have $\pdv{h_j}{p_j} < 0$. This means there must be some good $i$ such that $\pdv{h_i}{p_j} > 0$.
\end{proof}

If we want to examine possible substitution when budget is held fixed rather than when utility is held fixed, we have a similar definition using Marshallian demand instead of Hicksian demand.

\begin{definition*}[Gross substitutes]
We say that good $i$ is a \vocab{gross substitute} for good $j$ if
$$\pdv{x_i}{p_j} > 0.$$
\end{definition*}
That is, good $i$ is a gross substitute for good $j$ if an increase in the price of $j$ results in increased consumption of $i$ under a fixed budget. Notice that this property is not symmetric. For a counterexample, consider two goods $x$ and $y$ with a quasilinear utility function $U(a, b) = a + \log b$. We encourage this counterexample as an exercise to the reader, but in this case, $b$ is a gross substitute for $a$, but not vice versa.
\TODO{I'm not sure if this is mean}

\subsection*{Slutsky Equation with Two Goods}

If we have only two goods, we can derive Slutsky's equation directly for more intuition. We start with the utility-maximization problem 
$$
\max _{a, b \geq 0} u(a, b) \text { s.t. } p_{a} a+p_{b} b \leq y.
$$
We substitute the budget constraint to obtain the unconstrained univariate optimization problem 
$$
\max _{a \in\left[0, y / p_{a}\right]} u\left(a, \frac{y-p_{a} a}{p_{b}}\right).
$$
The first order condition is 
$$
u_{a}\left(a^{*}, b^{*}\right)-\frac{p_{a}}{p_{b}} u_{b}\left(a^{*}, b^{*}\right)=0.
$$
Together with the budget constraint 
$$
b^{*}=\frac{y-p_{a} a^{*}}{p_{b}},
$$
these two conditions give implicit solutions for Marshallian demands $a^*(p_a, p_b, y)$ and $b^*(p_a, p_b, y)$.
We differentiate the first order condition with respect to $p_a$ with the Implicit Function Theorem, yielding 
$$
-\frac{u_{b}}{p_{b}}+\left(u_{a a}-\frac{p_{a}}{p_{b}} u_{a b}\right) \frac{\partial a^{*}}{\partial p_{a}}+\left(u_{a b}-\frac{p_{a}}{p_{b}} u_{b b}\right) \frac{\partial b^{*}}{\partial p_{a}}=0.
$$
Substituting our expression for $b^*$ from the budget constraint yields 
$$
-\frac{u_{b}}{p_{b}}+\left(u_{a a}-\frac{p_{a}}{p_{b}} u_{a b}\right) \frac{\partial a^{*}}{\partial p_{a}}-\left(u_{a b}-\frac{p_{a}}{p_{b}} u_{b b}\right) \frac{1}{p_{b}}\left(a^{*}+p_{a} \frac{\partial a^{*}}{\partial p_{a}}\right)=0.
$$
We can rearrange these terms to get
$$
-\left[u_{a a}-2 \frac{p_{a}}{p_{b}} u_{a b}+\left(\frac{p_{a}}{p_{b}}\right)^{2} u_{b b}\right] \frac{\partial a^{*}}{\partial p_{a}}=-\frac{u_{b}}{p_{b}}-\frac{1}{p_{b}}\left(u_{a b}-\frac{p_{a}}{p_{b}} u_{b b}\right) a^{*}.
$$

What does all of this mean? The left hand side is the comparative static $\pdv{a^*}{p_a}$, multiplied by
$$
S \equiv -\left[u_{a a}-2 \frac{p_{a}}{p_{b}} u_{a b}+\left(\frac{p_{a}}{p_{b}}\right)^{2} u_{b b}\right],
$$
which is the value that must be positive in the second-order condition for a maximum. We see the substitution effect 
$$
-\frac{u_{b}}{S p_{b}}<0,
$$
which we can verify is the same as $\pdv{h_a}{p_a}$ if we set up the expenditure-minimization problem accordingly. Lastly, we see the income effect 
$$
-\frac{1}{S p_{b}}\left(u_{a b}-\frac{p_{a}}{p_{b}} u_{b b}\right) a^{*}
$$
which is equal to $-\frac{\partial a^{*}}{\partial y} a^{*}$ if we compute $\frac{\partial a^{*}}{\partial y}$ with the implicit function theorem.

\section{Income and Price Elasticity}
One common way to measure how income and prices affect demand is to use elasticities. These are unit invariant ways of seeing how given change affects demand for a good. To motivate the use of elasticities, we can use them to examine some intuition behind the income and price effects of goods. To start, we can differentiate our original budget constraint, 
$$y = \vec{p} \cdot \vec{x}$$
in order to identify a few important elements. Differentiating both sides with respect to $y$ yields\
\begin{align*}
    1 &=\sum_{i=1}^{n} p_{i} \frac{\partial x_{i}}{\partial y} \\
    &= \sum_{i=1}^{n} \frac{p_{i} x_{i}}{y} \frac{y \partial x_{i}}{x_{i} \partial y}.
\end{align*}


We now define two new terms. We define 
$$\alpha_i \equiv \frac{p_{i} x_{i}}{y}$$
to be the \vocab{budget share} of good $i$, which represents the fraction of our budget that we spend on good $i$. We define 
$$\varepsilon_y^i \equiv \frac{y \partial x_{i}}{x_{i} \partial y}$$
to be the \vocab{income elasticity of demand} for good $i$. Intuitively, this tells us the percentage change in demand for a small percentage change in income. You may have seen the elasticity of income in previous economic classes defined as,
\begin{align*}
    \varepsilon_y^i = \frac{\% \text{ change in demand}}{\% \text{ change in income}}
\end{align*}

How can we square these two definitions? We can start by rewriting our definition of the elasticity to be
\begin{align*}
    \varepsilon_y^i \equiv \frac{\partial x_i}{x_i}  \frac{y}{\partial y} = \frac{\partial x_i / x_i}{\partial y / y}
\end{align*}
We abuse the definition of the derivative slightly and treat it as a fraction even though this is not the case formally. Notice however that $\frac{\partial x_i}{x_i}$ is a small change in $x_i$ divided by current consumption of $x_i$, which is a small percentage change in $x_i$. Following the same logic, $\frac{\partial y}{y}$ is a small percentage change in income. So, our definition of income elasticity is in fact equivalent to the interpretatation as small percentage changes.

Another sometimes useful fact is that the income elasticity is definable in terms of log derivatives:
\begin{align*}
    \varepsilon_y^i = \frac{\partial x_i y}{x_i \partial y} = \frac{\partial \log x_i }{\partial \log y_i}
\end{align*}
In general, you can think of a change in the value of the logarithm of a value as a percentage change of said value. 

Notice the relationship between income elasticity and the income effect for a good: the income effect for a good $i$ is negative if and only if $\varepsilon_y^i > 0$, which makes sense given our intuitive understanding of elasticity.

Substituting these new terms into our previous equation, we have the relationship
$$
\sum_{i=1}^{n} \alpha_{i} \varepsilon_{y}^{i} = 1.
$$
The term on the left is a weighted average of the different $\varepsilon_{y}^{i}$ terms, since the some of all our $\alpha_i$ terms must be 1. Since the term on the left is in units of elasticity, we can almost think of this relationship as the idea that with a $1\%$ increase in budget, we can buy $1\%$ more ``stuff'' (not necessarily $1\%$ more quantity of consumption), where the composition of this ``stuff'' is determined by each good's budget share and income elasticity. \TODO{I'm not sure if this explanation makes any sense; I can remove it if it doesn't.} Notice that this relationship also tells us that it is impossible for every good in our option set to have a negative income elasticity; if our budget increases, we should not be buying less of everything. Later, we will see that this means that it is impossible for every good in our option set to be an inferior good.

We can derive a similar relationship to help us understand the substitution effect if we instead differentiate our original budget constraint with respect to the price of some good $p_j$. From the product rule, we get
$$
0=x_{j}+\sum_{i=1}^{n} p_{i} \frac{\partial x_{i}}{\partial p_{j}}.
$$
Multiplying by $\frac{p_j}{y}$ and rearranging yields 
$$
-\frac{p_{j} x_{j}}{y}=\sum_{i=1}^{n} \frac{p_{i} x_{i}}{y} \frac{p_{j} \partial x_{i}}{x_{i} \partial p_{j}}.
$$

We define 
$$\varepsilon_{p_j}^i \equiv \frac{p_j \partial x_{i}}{x_{i} \partial p_j}$$
as the \vocab{cross-price elasticity of demand}, which represents how sensitive our demand for good $i$ is in response to changes in the price of some other good $j$. When $i=j$, then this term is simply the \vocab{price elasticity of demand}. If $\varepsilon_{p_j}^i = 1$ always, then every $1\%$ increase in the price of good $j$ results in a $1\%$ increase in the consumption of good $i$.

Substituting our definitions for cross-price elasticity and budget share, we get the relationship
$$
-\alpha_{j}=\sum_{i=1}^{n} \alpha_{i} \varepsilon_{p_{j}}^{i}.
$$
This equation tells us that the weighted average of cross-price elasticities of all goods with respect to price changes for good $j$ is equal the negative budget share of good $j$. We will more closely examine the $\frac{\partial x_{i}}{\partial p_j}$ piece of our definition of cross-price elasticity later in the section on the general Slutsky equation.

\subsection*{Types of Goods}

We might already have some intuitive idea of what goods have a negative income effect (i.e. we consume more when we get richer) and what goods have a positive income effect (i.e.e we consume less when we get richer. We can further categorize these types of goods using our definition of income elasticity:
\begin{description}
\item[\vocab{Inferior goods}] have a negative income elasticity $\varepsilon_y^i < 0$, so their income effect is positive. This is equivalent to saying that $\frac{\partial x_i}{\partial y} < 0$. That is, we buy less of such goods as we become richer (e.g. cheap food). Recall from the previous section that it is possible for some or no goods to be inferior, but it is impossible for all goods to be inferior. 

Normally, the demand-curve for inferior goods is still downward-sloping; that is, we still have $\pdv{x_i}{p_i} < 0$. This is because even though we have positive income effect $-x_{i} \frac{\partial x_{i}}{\partial y}>0$, typically the negative substitution effect $\pdv{h_i}{p_i} < 0$ still dominates. Intuitively, an increase in the price of good $i$ has two effects for an inferior good: (1) it makes the consumer poorer, which influences consumption positively from the positive income effect of inferior goods, and (2) it makes the consumer substitute away to relatively cheaper options, which influences consumption negatively. If the second effect dominates, as is often the case, then the demand curve for an inferior good is still downward sloping.

However, it is theoretically possible for the former effect to dominate, causing demand to be \textit{upward-sloping}, or $\pdv{x_i}{p_i} > 0$. That is, when prices for good $i$ rise, the consumption of good $i$ rises. These goods are called \vocab{Giffen goods} and are a rare subset of inferior goods. This phenomenon occurs if 
$$
\bigg | x_{i} \frac{\partial x_{i}}{\partial y} \bigg | > \pdv{h_i}{p_i},
$$
which generally happens when 
$$x_{i} \frac{\partial x_{i}}{\partial y} \ll 0.$$
Intuitively, for the income effect to be so large, the good must be very inferior (i.e. $\varepsilon_y^i \ll 0$) and the budget share of $x_i$ must be very large, which is generally the case when the consumer is poor. A famous example of this phenomenon was presented by \href{https://www.ncbi.nlm.nih.gov/pmc/articles/PMC2964162/}{Jensen and Miller (2008)}. They found that subsidizing the price of rice in rural Hunan, China, resulted in decreased consumption of rice among poor households. Why? Poor households who used to spend a large fraction of their income on rice became effectively richer through this subsidy, so they no longer needed to buy as much rice since they could afford other foods in their diets.

\item[\vocab{Normal goods}] have positive income elasticity $\varepsilon_y^i > 0$, so their income effect is negative. This is equivalent to saying that $\frac{\partial x_i}{\partial y} > 0$. That is, we consume more of them when we become richer; most goods we typically think of fall under this category. The demand curve is also unambiguously downward-sloping, since both the income effect and the substitution effect are negative. By the same reasoning as before, some or all of the goods in our option set must be normal.

It makes sense to buy more of a good when we become richer, but the relative size of this increase might be bigger for some goods than others. A normal good is a \vocab{luxury} if $\varepsilon_y^i > 1$, so a $1\%$ increase in income results in more than a $1\%$ increase in consumption of that good. The budget share of these goods increases as the consumer becomes richer. For example, a consumer might spend no money on luxury cars initially, but they might spend a sizable amount of money on luxury cars when they become rich.

Alternatively a normal good is a \vocab{necessity} if $\varepsilon_y^i < 1$, so a $1\%$ increase in income results in an increase in consumption of that good by between $0\%$ and $1\%$. The budget share of these goods decreases as the consumer becomes richer. For example, Engel's law is the observation that the budget share for food decreases when income rises, so food is a necessity.

In the special case where $\varepsilon_y^i = 1$ for all goods in the option set, we say that preferences are \vocab{homothetic}. That is, budget shares of all goods are fixed, so a $1\%$ increase in income causes a $1\%$ increase in the consumption of all goods.

\end{description}


\section{Measuring Welfare Changes}
One of the challenges in dealing with utility in economics is that it is not directly observable. However, it would be nice to be able to quantify the change in utility from a change in prices. Fortunately, the expenditure function offers us a nice way to do so.

\subsection*{Money Metric Utility}
The money metric utility function offers us a way to measure utility changes in dollars, which is observable in the real world. In particular, it allows us a way to map changes in prices to a change in income. Suppose we have some utility function $u$ that we want to be able to measure in dollars. Take two utility levels, $\bar{u}_1, \bar{u}_2$ such that $\bar{u}_1 > \bar{u}_2$. Recall that for any set of prices, $\vec{r}$, we have
\begin{align*}
    e(\vec{r}, \bar{u}_1) > e(\vec{r}, \bar{u}_2)
\end{align*}
This tells us that $e(\vec{r}, \cdot)$ is a monotonic transformation in its second argument, so it can be applied to any valid utility function to create a new utility function. So, for any allocation $\vec{x}$, we can define the \vocab{money metric utility function} with reference prices $\vec{r}$ as,
\begin{align*}
    m(\vec{x}) = e(\vec{r}, u(\vec{x}))
\end{align*}
The idea here is that we can measure how much utility we have for a given allocation by examining what is the minimum amount that we would need to spend to achieve the utilty for that allocation. 

This gives us a natural way to compare utilities for two different sets of prices. Consider prices $\vec{p}$ and $\vec{q}$. For a fixed level of income $y$, we can compare their utilities through their respective value functions, $v(\vec{p}, y)$ and $v(\vec{q}, y)$. We can then convert these utility values into a money metric values to measure how much better (or worse) in dollars $\vec{p}$ is compared to $\vec{q}$
\begin{align*}
    m(v(\vec{p}, y)) - m(v(\vec{q}, y)) = e(\vec{r}, v(\vec{p}, y) - e(\vec{r}, v(\vec{q}, y))
\end{align*}
If the above expression is positive, then $\vec{p}$ is preferred to $\vec{q}$, and if it is negative, then $\vec{q}$ is preferred to $\vec{p}$. Importantly, these quantities are ones that we could in principle measure, by asking or observing how much money an individual requires at prices $\vec{r}$ to be indifferent with having prices $\vec{p}$ and income $y$. 

It is important to note, however, that while the money metric utility can be measured, it still does \emph{not} allow us to compare utilities across individuals.

\subsection*{Equivalent and compensating variation}
So far, we have used the money metric utility function to measure the differences between two sets of prices, $\vec{p}$ and $\vec{q}$, with respect to some set of reference prices $\vec{r}$. However, there is a question of how we select this reference price $\vec{r}$. Two reasonable prices we might pick are either $\vec{p}$ or $\vec{q}$. For the sake of this analysis, we will refer to $\vec{p}$ as the \emph{old} prices and $\vec{q}$ as the \emph{new} prices.

\subsubsection*{Equivalent Variation}
The \vocab{equivalent variation (EV)} tells us the welfare effect of a price change from $\vec{p}$ to $\vec{q}$ through an equivalent change in income at \emph{old} prices, $\vec{p}$. Mathematically, we are setting the reference price $\vec{r} = \vec{p}$,
\begin{align*}
    EV = e(\vec{p}, v(\vec{q}, y)) - e(\vec{p}, v(\vec{p}, y)) = e(\vec{p}, v(\vec{q}, y)) - y
\end{align*}
Intuitively, we can think of EV as how much you would be willing to pay to keep prices at $\vec{p}$ instead of $\vec{q}$. An easy way to remember this is that the equivalent variation tells you what would have been the \emph{equivalent} change in income for a given price change.

\subsubsection*{Compensating Variation}
The \vocab{compensating variation (CV)} tells us the welfare effect of a price change from $\vec{p}$ to $\vec{q}$ through an equivalent change in income at \emph{new} prices, $\vec{q}$. Mathematically, we are setting the reference price $\vec{r} = \vec{q}$,
\begin{align*}
    CV = e(\vec{q}, v(\vec{q}, y)) - e(\vec{q}, v(\vec{p}, y)) = y - e(\vec{q}, v(\vec{p}, y))
\end{align*}
Intuitively, we can think of CV as how much you would be willing to pay to change prices from $\vec{p}$ to $\vec{q}$. Or equivalently, it is the negative of how much you would need to be paid after a price change to make you just as happy as you were before. This second interpretation is useful for remembering the definition of compensating variation, as it tells you how much you need to be \emph{compensated} to be just as happy for a given price change. 

\subsubsection*{Hicksian Integral}
The equivalent and compensating variations have an interesting connection to the Hicksian demand function when we consider the change in only one price.  Suppose that we have a single price change, from $p$ to $q$.

First, we will examine the equivalent variation, which is
\begin{align*}
    EV = e(p, v(q, y)) - e(p, v(p, y)) = e(p, v(q, y)) - y
\end{align*}
However, we also have that $y = e(q, v(q, y))$. So we can rewrite the equivalent variation as,
\begin{align*}
    EV = e(p, v(q, y)) - e(q, v(q, y))
\end{align*}
Next, notice that for an arbitrary price $s$, we have by Shephard's Lemma\TODO{Add reference},
\begin{align*}
    \frac{\partial e(s, v(q, y))}{\partial s} = h(s, v(q, y))
\end{align*}
Where $h$ is the Hicksian demand for the single good. Then, by the fundamental theorem of calculus, we have
\begin{align*}
    EV &= e(p, v(q, y)) - e(q, v(q, y)) \\
    &= \int_q^p \frac{\partial}{\partial s} e(s, v(q, y)) \, ds \\
    &= \int_q^p h(s, v(q, y)) \, ds
\end{align*}

We can perform a similar process with the compensating variation to obtain,
\begin{align*}
    CV &= e(p, v(p, y)) - e(q, v(p, y)) \\
    &= \int_q^p \frac{\partial}{\partial s} e(s, v(p, y)) \, ds \\
    &= \int_q^p h(s, v(p, y)) \, ds
\end{align*}

Note that while we handled these in the single good case, it generalizes to multiple goods. A simple way to see this is to consider multiple goods, but only changing the price of one good. Then we obtain the same results as above except the integral is with respect to the price of that specific good, and the the Hicksian is the Hicksian demand of that specific good. 

\subsection*{Consumer Surplus}
You may be more familiar with another way of measuring changes in welfare: consumer surplus. In an introductory economics class, consumer surplus is presented as the the difference between how much consumers are willing to pay and the price they are charged, aggregated over all consumers. 

Suppose we faced a change in prices from $p$ to $q$ where $q < p$. Under the introductory economics interpretation, the change in consumer surplus ($\Delta CS$) would look something like this: 
\image[0.5]{plots/ch7_cs.png}

However, the problem with this approach relies is that it relies on there being many consumers each willing to buy one unit of the good for this line of logic to work. It might also make sense for an individual to have consumer surplus. Ideally, we would want a way to find the change in consumer surplus given an individual's marshallian demand curve. To do so, we can express the above chart with flipped axes so that we can write our demand function as the more familiar Marshallian demand, $x(p)$, 
\image[0.5]{plots/ch7_cs_flipped.png}

It now becomes clear that the change in \vocab{consumer surplus} is the area under the Marshallian demand curve from $q$ to $p$. We can express this as an integral to obtain,
\begin{align*}
    \Delta CS = \int_q^p x(s) \, ds
\end{align*}

This might look familiar because it is the same as the integral for EV and CV, except we integrate the Marshallian demand function rather than the Hicksian demand functions. This measure also works both for an individual, or an aggregate demand curve derived by summing many individual Marshallian demands. 

\subsection*{Comparing the welfare measures}
Now that we have three different measures of welfare changes, it is worth asking how they compare to each other. For simplicity, we will examine the case where the price of a good decreases from $p$ to $q$. 

Since we have expressed equivalent variation, compensating variation, and consumer surplus as integrals of Hicksians and Marshallians, we can determine their relative values by comparing the Hicksian and Marshallian demand functions. We will try to examine these functions graphically.

\subsubsection*{Normal good}
First, we will consider the case where the good is normal. This means that the income effect is positive for a price decrease, so the slope of the Marshallian demand curve is steeper than that of the Hicksian demand curve. We also know that $v(q, y) \geq v(p, y)$ because $q < p$, so the agent prefers price $q$. Since the good is normal, this implies that $h(s, v(q, y)) \geq h(s, v(p, y))$ for all prices $s$ because if we need to achieve a greater utility level, we will demand more of the normal good. Finally, recall the optimality conditions that $h(p, v(p, y)) = x(p)$ and $h(q, v(q, y)) = x(q)$. This gives us sufficient sufficient information to plot the functions as follows:
\image[0.75]{plots/ch7_ev_cv_cs_normal.png}

Based on this graphic, we have
\begin{align*}
    \begin{cases}
        EV &= A + B + C \\
        \Delta CS &= B + C \\
        CV &= C
    \end{cases}
\end{align*}

So, in the case of a normal good with a price decrease, we have
\begin{align*}
    EV \geq \Delta CS \geq CV
\end{align*}

The inequalities are reversed if we have a normal good with a price increase. It is worth noting the condition under which we have equality. In order for the thre quantities to be equal, we would need the slope of the Marshallian demand curve to be the same as the slope of the Hicksian demand curves. The only way this happens is if there is no income effect.

\subsubsection*{Inferior good}
Next, we will consider what would happen if the good is inferior. In this case, the slope of the Marshallian demand curve would be less steep than the the slope of the Hicksians since the income and substitution effects would move in opposite directions. We would still however need the points of intersection for the Hicksians and the Marshallian to remain the same. This would result in the following:
\image[0.75]{plots/ch7_ev_cv_cs_inferior.png}

This results in the following values:
\begin{align*}
    \begin{cases}
        EV &= F \\
        \Delta CS &= E + F \\
        CV &= D + E + F
    \end{cases}
\end{align*}

So, for an \emph{inferior good}, 
\begin{align*}
    EV \leq \Delta CS \leq CV
\end{align*}

Once again, if you have a price increase instead of a price decrease, the inequalities are reversed. 

Note however that regardless of whether we have a normal or inferior good, the change in consumer surplus is always between the equivalent and compensating variations. 

\section{Measuring Price Changes}

One important application of our study of the expenditure function is a metric for measuring price changes, or a \vocab{price index}. Prices for goods often change, but what would be a good way to aggregate different price changes across multiple goods into a single metric?

One appealing approach is to use the expenditure function. Suppose we start with utility $u_0$ when prices are at $\vec{p}^0$, and then prices change to $\vec{p}^1$. A natural metric might measure how much expenditure would need to change in order to maintain utility $u_0$ at this new price level $\vec{p}^1$. Mathematically, this definition would look like
$$\frac{e(\vec{p}^1, u_0)}{e(\vec{p}^0, u_0)}.$$
As another way to interpret this definition, if we had income $y_0$ when prices were $\vec{p}^0$, and then prices changed to $\vec{p}^1$ and our income changed to $\frac{e(\vec{p}^1, u_0)}{e(\vec{p}^0, u_0)}y_0$, then our utility would be the constant.

The definition above is wonderful in theory but cannot be measured in practice: since we cannot observe consumers' utilities, we also cannot observe their expenditure functions. However, one workaround is to recognize that 
$$e(\vec{p}, u) = \vec{p}\cdot \vec{h} (\vec{p}, u)$$
by definition, and that
$$\mathbf{x}(\mathbf{p}, y)=\mathbf{h}(\mathbf{p}, v(\mathbf{p}, y))$$
by duality. It is important to understand why the equation above holds: the utility-maximizing consumption given a budget (Marshallian demand) is the same as the expenditure-minimizing consumption to achieve this realized utility level (Hicksian demand). This observation is useful because although we cannot measure the expenditure function, we can measure Marshallian demand, since $\vec{x}, \vec{p}$, and $y$ are all observed.

Using these formulations, we can approximate 
$$\frac{e\left(\mathbf{p}^{1}, u_{0}\right)}{e\left(\mathbf{p}^{0}, u_{0}\right)} \approx \frac{\sum_{i=1}^{n} p_{i}^{1} x_{i}^{0}}{\sum_{i=1}^{n} p_{i}^{0} x_{i}^{0}},$$
where the approximation on the right-hand side is known as the \vocab{Laspeyres price index}. Notice where this approximation is not exact. The denominators are equal, since 
$$e\left(\mathbf{p}^{0}, u_{0}\right) = \sum_{i=1}^{n} p_{i}^{0} x_{i}^{0}$$
by definition of the expenditure function, but the numerators differ:
$$e\left(\mathbf{p}^{1}, u_{0}\right) \leq \sum_{i=1}^{n} p_{i}^{1} x_{i}^{0}.$$
Intuitively, this is because we know $\vec{x}^0$ achieves our minimum utility $u_0$, but purchasing this same bundle might no longer be the minimum expenditure to achieve this utility once prices change. Thus, 
$$\frac{e\left(\mathbf{p}^{1}, u_{0}\right)}{e\left(\mathbf{p}^{0}, u_{0}\right)} \leq \frac{\sum_{i=1}^{n} p_{i}^{1} x_{i}^{0}}{\sum_{i=1}^{n} p_{i}^{0} x_{i}^{0}},$$
so the Laspeyres price index is an overestimate of our theoretical ideal. If we had an income $y_0$ initially and multiplied it by the Laspeyres price index after the price change, we would be able to weakly increase our utility, since we can still afford our old consumption bundle, but we may be able to reoptimize and do even better.

A similar approach to the same problem is the \vocab{Paasche price index}
$$\frac{e\left(\mathbf{p}^{1}, u_{1}\right)}{e\left(\mathbf{p}^{0}, u_{1}\right)} \approx \frac{\sum_{i=1}^{n} p_{i}^{1} x_{i}^{1}}{\sum_{i=1}^{n} p_{i}^{0} x_{i}^{1}},$$
where we now weight prices by $\vec{x}^1$ instead of $\vec{x}^0$. The theoretical ideal on the left is a similar idea but uses the new utility level as the baseline rather than the old utility level. The Paasche price index on the right is now an underapproximation of the theoretical ideal on the left, since while the numerators are equal, the denominators differ as
$$e\left(\mathbf{p}^{0}, u_{1}\right) \leq \sum_{i=1}^{n} p_{i}^{0} x_{i}^{1}$$
by a similar reasoning to before. Intuitively, the Laspeyres price index measures how much a consumption bundle that was optimal in the base year costs now, whereas the Paasche price index measures how much a consumption bundle that is optimal now would have cost in the base year. 

\section*{Recap}
In this chapter, we discussed how we could derive the Hicksian demand function and the resulting expenditure function. This allowed us to derive the famous Slutsky equation, which decomposed the effects of a price change into a substitution effect and an income effect. We defined normal goods as those with a negative income effect in response to a price increase, while inferior goods have a positive income effect in response to a price increase. We also examined how we could use the expenditure function to derive a measurable utility function, as well as how to use that utility function to measure changes in welfare from price increases. Finally, we examined how the expenditure function could be used to construct an index of prices to see how prices have changed in the economy as a whole. 